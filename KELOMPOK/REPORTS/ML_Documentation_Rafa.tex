\documentclass[11pt,a4paper]{article}

% -------- Encoding & layout --------
\usepackage[utf8]{inputenc}
\usepackage[T1]{fontenc}
\usepackage[margin=2.5cm]{geometry}
\usepackage{parskip}
\usepackage{setspace}
\usepackage{float}
\usepackage{titlesec}
\usepackage{authblk}
\renewcommand\Authfont{\large\bfseries}
\renewcommand\Affilfont{\small\normalfont}
\setlength{\affilsep}{0.4em}
\renewcommand\Authsep{ \quad}
\renewcommand\Authand{ \quad}
\renewcommand\Authands{ \quad}

% -------- Math & figures --------
\usepackage{amsmath,amssymb}
\usepackage{graphicx}
\usepackage{booktabs}
\usepackage{enumitem}
\usepackage{caption}
\usepackage{xcolor}
\usepackage{listings}

% -------- Hyperref --------
\usepackage{hyperref}
\hypersetup{
    colorlinks=true,
    linkcolor=blue!60!black,
    urlcolor=blue!60!black,
    citecolor=blue!60!black,
    pdfauthor={Rafa Al Razzak et al.},
    pdftitle={Dokumentasi Kode: 18 Studi Kasus Machine Learning},
    pdfsubject={Regresi, Klasifikasi, Klastering, Association Rules},
    pdfkeywords={Machine Learning, Scikit-learn, Python, Documentation}
}

% -------- Section spacing --------
\titlespacing*{\section}{0pt}{6pt}{2pt}
\titlespacing*{\subsection}{0pt}{4pt}{2pt}
\titlespacing*{\subsubsection}{0pt}{3pt}{1pt}

% -------- Listings (Python) --------
\lstdefinestyle{pythonstyle}{
    language=Python,
    basicstyle=\ttfamily\small,
    showstringspaces=false,
    breaklines=true,
    keywordstyle=\color{blue!70!black},
    stringstyle=\color{teal!60!black},
    commentstyle=\color{gray!70},
    frame=single,
    rulecolor=\color{black!20},
    frameround=ffff,
    framesep=6pt,
    numbers=left,
    numberstyle=\tiny\color{gray},
    stepnumber=1,
    numbersep=8pt,
    columns=fullflexible
}
\lstset{style=pythonstyle}
\captionsetup[lstlisting]{labelfont=bf,textfont=bf}

% -------- APA references (choose one) --------
% Option A (recommended): biblatex-apa (requires biber)
% \usepackage[style=apa,natbib=true,backend=biber]{biblatex}
% \DeclareLanguageMapping{english}{english-apa}
% \addbibresource{references.bib}

% Option B: Manual references (no extra tooling)

% -------- Meta --------
\title{\textbf{Algoritma Machine Learning}\\
\large (Regression, Classification, Clustering, Association Rules)}
\author[1]{Rafa Al Razzak}
\author[2]{Oryza Ayunda Putri}
\author[3]{Adelia Juliani}
\affil[1,2,3]{Program Studi Teknik Informatika, STT Terpadu Nurul Fikri, Depok, Indonesia\\
\texttt{0110224155@student.nurulfikri.ac.id} \quad
\texttt{0110224030@student.nurulfikri.ac.id} \quad
\texttt{0110224113@student.nurulfikri.ac.id}}
\date{}

\begin{document}
    \maketitle
    \onehalfspacing
    \begin{abstract}
        Dokumen ini adalah \emph{code documentation} yang terstruktur untuk 18 studi kasus Machine Learning (5 regresi, 5 klasifikasi, 5 klastering, 3 association rules). Seluruh data sintetis (reproducible). Setiap blok kode dilengkapi tujuan, langkah, dan catatan implementasi. Visualisasi ringkas pada notebook (matplotlib).
    \end{abstract}

    \tableofcontents
    \vspace{6pt}\hrule\vspace{6pt}


    \section{Cara Pakai}
    \begin{enumerate}[leftmargin=1.2em]
        \item \textbf{Buka} notebook: \texttt{ML\_Learning\_Types\_Overview.ipynb} lalu \texttt{ML\_18\_Cases\_Demo.ipynb}.
        \item \textbf{Jalankan berurutan:} Setup $\rightarrow$ Regresi $\rightarrow$ Klasifikasi $\rightarrow$ Klastering $\rightarrow$ Association.
        \item \textbf{Eksperimen:} Ubah jumlah data (\texttt{n}) dan hiperparameter model; semua deterministik (\texttt{np.random.seed(42)}).
    \end{enumerate}


    \section{Definisi Singkat (Point 1 Tugas)}

    \subsection{Supervised Learning}
    Belajar pemetaan input $\to$ output dari \textbf{data berlabel}. Tahap: (1) \textit{framing} masalah, (2) koleksi/label data, (3) train/val split, (4) pelatihan model, (5) evaluasi (mis. R$^2$, akurasi), (6) \textit{deployment}, (7) \textit{monitoring}.

    \subsection{Unsupervised Learning}
    Menemukan struktur dari \textbf{data tak berlabel}. Tahap: (1) framing, (2) koleksi, (3) skala/FE, (4) algoritma (KMeans/DBSCAN), (5) validasi internal (silhouette), (6) interpretasi, (7) pemanfaatan.

    \subsection{Reinforcement Learning}
    Agen belajar melalui interaksi lingkungan untuk memaksimalkan \textbf{reward} kumulatif. Tahap: (1) definisi MDP, (2) pilih pendekatan kebijakan/nilai, (3) eksplorasi, (4) evaluasi/perbaikan kebijakan, (5) iterasi hingga konvergen, (6) deployment dan monitoring.


    \section{18 Kasus (Point 2 Tugas)}
    Lihat notebook \texttt{ML\_18\_Cases\_Demo.ipynb} untuk kode lengkap, metrik, dan ringkasan visual.


    \section{Sistematika Laporan (Point 3 Tugas)}

    \subsection{Struktur}
    \begin{enumerate}[leftmargin=1.2em]
        \item Pendahuluan \quad (latar belakang, tujuan)
        \item Dasar Teori \quad (definisi SL/UL/RL)
        \item Metodologi \quad (data, pipeline, metrik)
        \item Hasil \& Pembahasan \quad (tabel/plot per kasus)
        \item Kesimpulan \& Saran
        \item Referensi (format APA)
    \end{enumerate}

    \subsection{Contoh Sitasi APA (manual)}
    \begin{itemize}[leftmargin=1.2em]
        \item Breiman, L. (2001). Random forests. \textit{Machine Learning}, 45(1), 5–32.
        \item Cortes, C., \& Vapnik, V. (1995). Support-vector networks. \textit{Machine Learning}, 20, 273–297.
        \item MacQueen, J. (1967). Some methods for classification and analysis of multivariate observations. In \textit{Proc. 5th Berkeley Symposium} (pp. 281–297).
        \item Ester, M., Kriegel, H.-P., Sander, J., \& Xu, X. (1996). A density-based algorithm for discovering clusters. In \textit{KDD} (pp. 226–231).
        \item Tibshirani, R. (1996). Regression shrinkage and selection via the lasso. \textit{JRSS B}, 58(1), 267–288.
        \item Hoerl, A. E., \& Kennard, R. W. (1970). Ridge regression. \textit{Technometrics}, 12(1), 55–67.
    \end{itemize}

% If using biblatex:
% \printbibliography

\end{document}
